\documentclass[12pt]{article}
\usepackage{chemfig} % chemical structures
\usepackage[T1]{fontenc} % font encoding for non-English characters
\usepackage[margin = 0 in, paperwidth = 3 in, paperheight = 1.5 in]{geometry} % margins
\usepackage{graphicx} % figures
\usepackage[utf8]{inputenc} % input encoding for unicode characters
\usepackage{sansmathfonts} % sans serif math fonts
\usepackage{tikz} % image positioning
\usepackage{xcolor} % colors

% other options
    \graphicspath{{./images}}
    \usetikzlibrary{shapes.geometric, arrows, positioning}
    \renewcommand{\familydefault}{\sfdefault}
    \renewcommand{\footnoterule}{}
    \let\thefootnote\relax
	\setlength{\parindent}{0pt}
    \pagenumbering{gobble}

\begin{document}

\def\h_length{1.5}
\def\bb_length{1.5}
\def\vert_length{1.5}

{\centering \small \begin{tikzpicture}[node distance = 0.25 cm, inner sep = 0 cm]
    \node {
        \chemfig[atom sep = 20 pt, double bond sep = 3 pt, cram width = 4 pt]{
            R
            -[:-30,\bb_length]C
            (=[6,\bb_length]O)
            -[:30,\bb_length]N
            (-[2,\h_length]H)
            -[@{f, 0.5}:-30,\bb_length]{{C\alpha}}
            (<[:-60,\h_length]{{H\alpha}})
            (
                <:[@{x, 0.5}:-120,\vert_length]{{C\beta}}
                -[:-60,\bb_length]{{C\gamma}}
            )
            -[@{y, 0.5}:30,\bb_length]C
            (=[2,\bb_length]O)
            -[:-30,\bb_length]N
            (-[6,\h_length]H)
            -[:30,\bb_length]R
        }
        \chemmove{
            \draw (f) .. controls +(45:0.5 cm) and +(-105:0.5 cm) .. (f);
            \node [above = of f] {$\phi$};
            \draw (y) .. controls +(105:0.5 cm) and +(-45:0.5 cm) .. (y);
            \node [above = of y] {$\psi$};
            \draw (x) .. controls +(-45:0.5 cm) and +(165:0.5 cm) .. (x);
            \node [left = of x] {$\chi_1$};
        }
    };
\end{tikzpicture} \par}

\end{document}
